\documentclass[]{BJTU-thesis}

%%%%%%%%%%%%%%%%%%%引入一些包%%%%%%%%%%%%%%%%%%%%%55
\usepackage[ruled,linesnumbered]{algorithm2e}
\usepackage{bicaption}
\usepackage{emptypage}
\usepackage{pdfpages}
% \usepackage{enumerate}
%\usepackage{subfig}
%\usepackage{subfig}
%   \setromanfont{Times New Roman}
 \usepackage{fontspec}
 \newfontfamily\myfont{times.ttf}
% \setromanfont[
% BoldFont=timesbd.ttf,
% ItalicFont=timesi.ttf,
% BoldItalicFont=timesbi.ttf,
% ]{times.ttf}
% 
% \setmainfont{Times New Roman}
\renewcommand\labelenumi{(\theenumi)}
\renewcommand{\algorithmcfname}{算法}

\DeclareMathOperator*{\argmin}{argmin}
\DeclareMathOperator*{\argmax}{argmax}

\captionsetup[figure][bi-first]{name=图}
\captionsetup[figure][bi-second]{name=Figure}

\captionsetup[table][bi-first]{name=表}
\captionsetup[table][bi-second]{name=Table}

%%%%%%%%%%%%%%定义一些东西%%%%%%%%%%%%%%%%%%%%%

%\newcommand{\E}{\mathbb{E}}


%\theoremstyle{plain} \theorembodyfont{\kai\rmfamily}
%\theoremheaderfont{\hei\rmfamily}\theoremseparator{.  }

\newtheorem{theorem}{定理}[section]
%\newtheorem{Theorem}{定理}[section]
\newtheorem{lemma}[theorem]{引理}
\newtheorem{corollary}[theorem]{推论}
%\newtheorem{corollary}[theorem]{推论}
\newtheorem{assumption}{假设}[chapter]
%		\newtheorem{Corollary}[Theorem]{推论}
%\newtheorem{Remark}[Theorem]{注}
\newtheorem{example}[theorem]{例}
\newtheorem{definition}[theorem]{定义}
%\newtheorem{Construction}[Theorem]{构造}
%		\def\binom#1#2{{#1\choose#2}}

\newcommand{\E}{\mathbb{E}}
\newcommand{\huaf}{\mathcal{F}}
\newcommand{\huao}{\mathcal{O}}
\newcommand\numberthis{\addtocounter{equation}{1}\tag{\theequation}}
\newcommand{\mne}{ \mathbf{MNE}}
\newcommand{\mnbe}{ \mathbf{MNBE}}

\SetKwInput{KwIn}{输入}
\SetKwInput{KwOut}{输出}


\graphicspath{{./figure/}}

%%%%%%%%%%%%%%%填写封面信息%%%%%%%%%%%%%%%%%%%%
\author{徐文龙}
\studentNumber{20140094}
\advisor{赵耀 }
 
\advisorTitle{教授}
 
\degreeType{计算机与信息技术}
\major{软件工程}
\researchArea{自然语言处理}
\title{半监督情感分类}
\englishtitle{Semi-supervised Emotion Classification}
%%%%%%%%%%%%%%%%%%%%%%%%%%%%%%%%%%%%%%%%%%%%%%
%\setmainfont{Times New Roman}

%\makeindex
\makeindex[name=class,options=-s mystyle,title=分类索引,columns=2]
\makeindex[name=author,options=-s mystyle,title=著者索引,columns=2]
\makeindex[name=keyword,options=-s mystyle,title=关键词索引,columns=2]
\begin{document}

\makecover
\makeAuthorization
\includepdf[pages=-]{chapters/empty.pdf}
\makeInfo
%\include{chapters/thanks}
\include{chapters/thanksshort}




% 
\begin{abstract}



    基于现代信息技术所变革出的以客户为中心的C2M(Customer to Manufacturer,客户到工厂)商业模式,以其更好的满足新时代的“小批量、多批次”的制造需求,给企业带来巨大成功的同时,如何高效获取用户需求成为了新的难题。
    传统的用户调研方式,耗时、费力、效果也并不理想。为了快速获得客户的需求,我们从电商平台的用户评论出发,通过模型挖掘用户评论中的情感信息,进行情感分析、观点抽取,获得用户对产品的反馈结果,进而生成C2M分析报告。
    实验结果表明,基于T5的预训练生成模型在进行用户评论的三元组(情感方面、情感词、情感极性时)抽取时,具有很好的性能。




    \vspace{16pt}

    \noindent
    \keywords{这里是关键词; 关键词2;}



\end{abstract}
\include{chapters/englishabstract}




%%%\include{chapters/preface}
\tableofcontents

\includepdf[pages=-]{chapters/empty.pdf} %%%%%注意,这里是在目录后加一个空白页,如果不需要空白页,请注释掉这一行

%%%%%注意,这里如果不需要空白页,请注释掉这一行↑





\newpage\pagenumbering{arabic}
%% 引言

\chapter{引言}




\section{研究背景}



\begin{figure}[h]
	\centering
	\includegraphics[width=3in,height=2in]{rlp}
	% 	\caption{  PSI transformation for RNN}
	\bicaption{  序列决策问题示例}{Illustration for sequential decision making problem}
	\label{fig:chap1 rlp}
\end{figure}






\section{本文的结构安排}
文本的结构安排如下:



\section{参考文献示例}
参考文献bib文件放在reference文件夹中,在主文件demo中引用了bib文件,在所有chap中都可以直接cite。比如\cite{silver2016mastering}

\section{算法框示例}

\begin{algorithm}[h]
	\caption{ 优化算法}
	\label{gtdmc:alg:minmax}
	\KwIn{ 迭代次数$ T$,   待学习参数初始值$x_1$ 、 $y_1$ , 学习率 $\alpha$ }
	\For{$ t = 1, \dots, T $}{
	更新参数:

	$ y_{t+1} = \mathcal{P}_{\mathcal{X}_y}\left(y_t + \alpha_t(\hat{b}_t - \hat{A}_t x_t -\hat{M}_ty_t)\right)  $

	$ x_{t+1} = \mathcal{P}_{\mathcal{X}_x}\left(x_t + \alpha_t\hat{A}_t^\top y_t\right) $
	}


	\KwOut{ 		$ \quad 	 \tilde{x}_T = \frac{\sum_{t=1}^{T}\alpha_t x_t}{\sum_{t=1}^{T}\alpha_t}  \qquad \tilde{y}_T = \frac{\sum_{t=1}^{T}\alpha_t y_t}{\sum_{t=1}^{T}\alpha_t} $	 }


\end{algorithm}

\section{图片示例}

%表格
\begin{table}
	\begin{tabular}{l c r}
		单元格1 & 单元格2 & 单元格3 \\
		单元格4 & 单元格5 & 单元格6 \\
		单元格7 & 单元格8 & 单元格9
	\end{tabular}
	\caption{表格标题 }
\end{table}

\begin{figure}[h]
	\centering
	\subfloat[a]{
		\label{figa}
		\includegraphics[width=1.8in,height=1.5in]{a.png}}
	\subfloat[b]{
		\label{figc}
		\includegraphics[width=1.8in,height=1.5in]{a.png}}
	\subfloat[c ]{
		\label{fige}
		\includegraphics[width=1.8in,height=1.5in]{a.png}}


	\subfloat[d ]{
		\label{figb}
		\includegraphics[width=1.8in,height=1.5in]{a.png}}
	\label{Fig1}
	\subfloat[e ]{
		\label{figd}
		\includegraphics[width=1.8in,height=1.5in]{a.png}}
	\subfloat[f ]{
		\label{figf}
		\includegraphics[width=1.8in,height=1.5in]{a.png}}
	\bicaption{你好,世界}{Hello world }
	\label{fig1}
\end{figure}

%%%% \include{chapters/background}

\include{chapters/summary}

\bibliography{reference/ref}
%% 附录
%\appendix
% \include{chapters/appendix}	
%%% 索引
%\chapter*{索引}
%\pagestyle{fancy}
%\addcontentsline{toc}{chapter}{索引}
%678
%\printindex[class]
%\addcontentsline{toc}{section}{分类索引}
%456
%\printindex[author]
%\addcontentsline{toc}{section}{著者索引}
%345
%\printindex[keyword]
%\addcontentsline{toc}{section}{关键词索引}
%123



\include{chapters/research}
%%% 独创性声明	
\include{chapters/declaration}
%% 学位论文数据集
%\include{chapters/dataset}
\includepdf[pages=-]{chapters/empty.pdf}
\includepdf[pages=-]{chapters/dataset.pdf}

%1232

\end{document}